%! TEX root = ../main.tex
\documentclass[dvipdfmx,12pt,beamer]{standalone}
\ifstandalone
	%%%% 和文用(よくわかってないです) %%%%
\usepackage{bxdpx-beamer} %ナビゲーションシンボル(?)を機能させる
\usepackage{pxjahyper} %しおりの日本語対応
\usepackage{minijs} %和文メトリックの調整
\usepackage{latexsym}
\usepackage[deluxe,expert]{otf}
\renewcommand{\kanjifamilydefault}{\gtdefault} %和文規定をゴシックに(変えたければrenewcommand で調整)
%%%%%%%%%%

%%%% TikZ %%%%
\usepackage{tikz}
\usetikzlibrary{calc,decorations.pathreplacing,quotes,positioning,shapes,fit,arrows,backgrounds,tikzmark}
%%%%%%%%

%%%% ファイル分割用のパッケージ %%%%
\usepackage{standalone}
\usepackage{import}
%%%%%%%%

%%%% スライドの見た目 %%%%%
\usetheme{metropolis}
\usepackage{xcolor}
%\usefonttheme{professionalfonts}
\setbeamercolor{alerted text}{fg=mLightBrown!90!black}
\setbeamersize{text margin left=.75zw, text margin right=.75zw}
\setbeamertemplate{section in toc}[ball unnumbered]
\setbeamertemplate{subsection in toc}[square]
%%%%%%%%%%

%%%%% フォント基本設定 %%%%%
\usefonttheme{professionalfonts}
\usepackage[T1]{fontenc}%8bit フォント
%\usepackage{textcomp}%欧文フォントの追加
\usepackage[utf8]{inputenc}%文字コードをUTF-8
\usepackage{txfonts}%数式・英文ローマン体を Lxfont にする
%\usepackage{mathpazo}
\usepackage{bm}%数式太字
%%%%%%%%%%

%%%% その他ほしかったもの%%%%
\usepackage{amsmath,amssymb}
\usepackage{amsthm}
\usepackage{algorithm, algpseudocode}
\usepackage{enumitem}
\setlist[itemize]{label=\textbullet}

\newcommand{\func}[1]{\ensuremath\mathrm{#1}}
\newcommand{\Scatter}{\text{\scshape Scatter}}
\newcommand{\Gather}{\text{\textsc{Gather}}}
\newcommand{\Propagate}{\func{\text{\textsc{Propagate}}}}
\newcommand{\ShiftAnd}{\textbf{Shift-And}}
\newcommand{\BNDM}{\textbf{BNDM}}
\providecommand{\as}{\textasteriskcentered}
\providecommand{\pl}{\text{+}}
%%%%%%%%

%%%% Standaloneかどうかでimportのパスを切り替える %%%%
\providecommand{\ImportStandalone}[3]{
	\IfStandalone{
		\import{#2}{#3}
		}{
		\import{#1#2}{#3}
	}
}
%%%%%%%%

\title{Extended string matching}
\institute{北海道大学 情報知識ネットワーク研究室 M2}
\author{光吉 健汰}

\fi	

\begin{document}
%\begin{frame}{諸定義}
%	\begin{description}
%		\item[ワードマシンモデル] $1W$
%	\end{description}
%\end{frame}

\begin{frame}{有限オートマトン}
	\begin{block}{非決定性有限オートマトン(Nondeterministic Finite Automaton; NFA)}
		以下の要素からなる有限オートマトンの一種
		\begin{columns}
			\begin{column}{.49\textwidth}
				\begin{itemize}
					\item 状態集合$Q$
					\item 入力アルファベット$\Sigma$
					\item 遷移関数$\delta : (\Sigma \cup \epsilon) \rightarrow 2^Q$
					\item 開始状態 $S \subset Q$
					\item 受理状態 $T \subset Q$
				\end{itemize}
			\end{column}
			\begin{column}{.5\textwidth}
				\begin{tikzpicture}[state/.style={circle, draw, minimum size=.5cm}, node distance=.5cm]
					\node(a0)[state] at (0,0) {$0$};
					\node(a1)[state, right =of a0] {$1$};
					\node(a2)[state, right =of a1] {$2$};
					\node(a3)[state, double, double distance=.8mm, right =of a2] {$3$};

					\draw[->] (-.5, .5) to (a0);
					\draw[->] (a0) to node[midway, above] {$a$} (a1);
					\draw[->] (a1) to node[midway, above] {$b$} (a2);
					\draw[->] (a2) to node[midway, above] {$c$} (a3);
					\draw[->] (a1) to [out=60, in=120, looseness=1] node[midway, above] {$d$} (a3);
				\end{tikzpicture}
			\end{column}
		\end{columns}
	\end{block}

	\begin{itemize}
		\item 文字列を受け取ることで、 受理できるかを判定することができる。
	\end{itemize}
\end{frame}
\end{document}
