%! TEX root = ../main.tex
\documentclass[dvipdfmx,12pt,beamer]{standalone}
\ifstandalone
	%%%% 和文用(よくわかってないです) %%%%
\usepackage{bxdpx-beamer} %ナビゲーションシンボル(?)を機能させる
\usepackage{pxjahyper} %しおりの日本語対応
\usepackage{minijs} %和文メトリックの調整
\usepackage{latexsym}
\usepackage[deluxe,expert]{otf}
\renewcommand{\kanjifamilydefault}{\gtdefault} %和文規定をゴシックに(変えたければrenewcommand で調整)
%%%%%%%%%%

%%%% TikZ %%%%
\usepackage{tikz}
\usetikzlibrary{calc,decorations.pathreplacing,quotes,positioning,shapes,fit,arrows,backgrounds,tikzmark}
%%%%%%%%

%%%% ファイル分割用のパッケージ %%%%
\usepackage{standalone}
\usepackage{import}
%%%%%%%%

%%%% スライドの見た目 %%%%%
\usetheme{metropolis}
\usepackage{xcolor}
%\usefonttheme{professionalfonts}
\setbeamercolor{alerted text}{fg=mLightBrown!90!black}
\setbeamersize{text margin left=.75zw, text margin right=.75zw}
\setbeamertemplate{section in toc}[ball unnumbered]
\setbeamertemplate{subsection in toc}[square]
%%%%%%%%%%

%%%%% フォント基本設定 %%%%%
\usefonttheme{professionalfonts}
\usepackage[T1]{fontenc}%8bit フォント
%\usepackage{textcomp}%欧文フォントの追加
\usepackage[utf8]{inputenc}%文字コードをUTF-8
\usepackage{txfonts}%数式・英文ローマン体を Lxfont にする
%\usepackage{mathpazo}
\usepackage{bm}%数式太字
%%%%%%%%%%

%%%% その他ほしかったもの%%%%
\usepackage{amsmath,amssymb}
\usepackage{amsthm}
\usepackage{algorithm, algpseudocode}
\usepackage{enumitem}
\setlist[itemize]{label=\textbullet}

\newcommand{\func}[1]{\ensuremath\mathrm{#1}}
\newcommand{\Scatter}{\text{\scshape Scatter}}
\newcommand{\Gather}{\text{\textsc{Gather}}}
\newcommand{\Propagate}{\func{\text{\textsc{Propagate}}}}
\newcommand{\ShiftAnd}{\textbf{Shift-And}}
\newcommand{\BNDM}{\textbf{BNDM}}
\providecommand{\as}{\textasteriskcentered}
\providecommand{\pl}{\text{+}}
%%%%%%%%

%%%% Standaloneかどうかでimportのパスを切り替える %%%%
\providecommand{\ImportStandalone}[3]{
	\IfStandalone{
		\import{#2}{#3}
		}{
		\import{#1#2}{#3}
	}
}
%%%%%%%%

\title{Extended string matching}
\institute{北海道大学 情報知識ネットワーク研究室 M2}
\author{光吉 健汰}

\fi	

\begin{document}
\begin{frame}{有界な文字の繰り返し}
	\begin{block}{有界な文字の繰り返し}
		
		文字xの$a$以上、$b$以下の繰り返しを表す記号$x(a, b)$
		
		例:$\text{x}(1,3) = \{\text{x},\text{xx},\text{xxx}\}$
	\end{block}
\end{frame}
\begin{frame}{基本アイデア}
	\begin{itemize}
		\item NFAを考える。
		\item $b$個の状態を用意し、xで遷移する。
		\item 繰り返しに対応するそれぞれの状態は、その後端の状態へ$\epsilon$遷移を持つ
	\end{itemize}
	\begin{block}{$\epsilon$遷移}
		NFA中で空文字を受け取る遷移
	\end{block}
	\begin{tikzpicture}[state/.style={circle, draw, minimum size=.7cm}, node distance=0cm]
		\node(chr0) [rectangle, draw] at (0,0) {$0$};
		\node(chr) [left =of chr0] {$CHR[\alpha]$};
		\foreach \x [count=\i from 0, remember=\i as \prev (initially 0)] in {1,0,1,0,0,0,0,0,0,1,0,1,0,0,0} {
			\node(chr\i) [rectangle, draw, right =of chr\prev] {$\x$};
		}
		\node(chr0a) [rectangle, draw] at (0,-.7) {$0$};
		\node(chra) [left =of chr0a] {$CHR[\alpha]$};
		\foreach \x [count=\i from 0, remember=\i as \prev (initially 0)] in {1,0,1,0,0,0,0,0,0,1,0,1,0,0,0} {
			\node(chr\i a) [rectangle, draw, right =of chr\prev a] {$\x$};
		}
	\end{tikzpicture}
\end{frame}
\end{document}
