%! TEX root = ../main.tex
\documentclass[dvipdfmx,12pt,beamer]{standalone}
\ifstandalone
	%%%% 和文用(よくわかってないです) %%%%
\usepackage{bxdpx-beamer} %ナビゲーションシンボル(?)を機能させる
\usepackage{pxjahyper} %しおりの日本語対応
\usepackage{minijs} %和文メトリックの調整
\usepackage{latexsym}
\usepackage[deluxe,expert]{otf}
\renewcommand{\kanjifamilydefault}{\gtdefault} %和文規定をゴシックに(変えたければrenewcommand で調整)
%%%%%%%%%%

%%%% TikZ %%%%
\usepackage{tikz}
\usetikzlibrary{calc,decorations.pathreplacing,quotes,positioning,shapes,fit,arrows,backgrounds,tikzmark}
%%%%%%%%

%%%% ファイル分割用のパッケージ %%%%
\usepackage{standalone}
\usepackage{import}
%%%%%%%%

%%%% スライドの見た目 %%%%%
\usetheme{metropolis}
\usepackage{xcolor}
%\usefonttheme{professionalfonts}
\setbeamercolor{alerted text}{fg=mLightBrown!90!black}
\setbeamersize{text margin left=.75zw, text margin right=.75zw}
\setbeamertemplate{section in toc}[ball unnumbered]
\setbeamertemplate{subsection in toc}[square]
%%%%%%%%%%

%%%%% フォント基本設定 %%%%%
\usefonttheme{professionalfonts}
\usepackage[T1]{fontenc}%8bit フォント
%\usepackage{textcomp}%欧文フォントの追加
\usepackage[utf8]{inputenc}%文字コードをUTF-8
\usepackage{txfonts}%数式・英文ローマン体を Lxfont にする
%\usepackage{mathpazo}
\usepackage{bm}%数式太字
%%%%%%%%%%

%%%% その他ほしかったもの%%%%
\usepackage{amsmath,amssymb}
\usepackage{amsthm}
\usepackage{algorithm, algpseudocode}
\usepackage{enumitem}
\setlist[itemize]{label=\textbullet}

\newcommand{\func}[1]{\ensuremath\mathrm{#1}}
\newcommand{\Scatter}{\text{\scshape Scatter}}
\newcommand{\Gather}{\text{\textsc{Gather}}}
\newcommand{\Propagate}{\func{\text{\textsc{Propagate}}}}
\newcommand{\ShiftAnd}{\textbf{Shift-And}}
\newcommand{\BNDM}{\textbf{BNDM}}
\providecommand{\as}{\textasteriskcentered}
\providecommand{\pl}{\text{+}}
%%%%%%%%

%%%% Standaloneかどうかでimportのパスを切り替える %%%%
\providecommand{\ImportStandalone}[3]{
	\IfStandalone{
		\import{#2}{#3}
		}{
		\import{#1#2}{#3}
	}
}
%%%%%%%%

\title{Extended string matching}
\institute{北海道大学 情報知識ネットワーク研究室 M2}
\author{光吉 健汰}

\fi	

\begin{document}
\begin{frame}{有界な文字の繰り返し}
	\begin{block}{有界な文字の繰り返し}
		
		文字xの$a$以上、$b$以下の繰り返しを表す記号$x(a, b)$
		
		例:$\text{x}(1,3) = \{\text{x},\text{xx},\text{xxx}\}$
	\end{block}
\end{frame}
\begin{frame}{有界な文字の繰り返し:基本アイデア}
	\begin{itemize}
		\item NFAを考える。
		\item $b$個の状態を用意し、xで遷移する。
		\item これらのうち始端の状態は、$b-a$個の状態へ$\epsilon$遷移を持つ。
		\item $\epsilon$遷移を持つ状態は、アクティブかどうかを\\伝搬させなければならない。
	\end{itemize}
	\begin{block}{$\epsilon$遷移}
		NFA中で空文字を受け取る遷移
	\end{block}
	例 $a(1, 3)$

	\begin{tikzpicture}[state/.style={circle, draw, minimum size=.7cm}, node distance=1cm]
		\node(a0) [state] at (0, 0) {$0$};
		\node(a1) [state, right =of a0]  {$1$};
		\node(a2) [state, right =of a1]  {$2$};
		\node(a3) [state, right =of a2]  {$3$};
		\draw[->] (a0) to node[midway, above] {a} (a1);
		\draw[->] (a1) to node[midway, above] {a} (a2);
		\draw[->] (a2) to node[midway, above] {a} (a3);
		\draw[dashed, ->] (a0) [bend left, looseness=2] to node[midway, above] {$\epsilon$} (a1);
		\draw[dashed, ->] (a0) [bend left] to node[midway, above] {$\epsilon$} (a2);
	\end{tikzpicture}
\end{frame}

\begin{frame}{有界な文字の繰り返し:$\ShiftAnd$への適用}
	有界な文字の繰り返しの場合、$\ShiftAnd$と$\BNDM$で行う改良は\\同じなので、$\ShiftAnd$について考える。

	状態を管理するビットマスク$D$について、$epsilon$遷移を考える。

	この遷移は、$b-a$ビットの区間に$1$を立てるような処理となる。

	$\rightarrow$ 引き算の\alert{繰り下がり}によって区間の更新を行う。

	例 $35 - 8$の繰り下がりの様子

	\centering
	\begin{tikzpicture}[state/.style={circle, draw, minimum size=.7cm}, node distance=0cm]
		\node(chr0) [rectangle, draw] at (0,0) {$1$};
		\node(chr) [left =of chr0] {$35$};
		\foreach \x [count=\i from 1, remember=\i as \prev (initially 0)] in {0,0,0,1,1} {
			\node(chr\i) [rectangle, draw, right =of chr\prev] {$\x$};
		}
		\node(chr0a) [rectangle, draw] at (0,-.7) {$0$};
		\node(chra) [left =of chr0a] {$- 8$};
		\foreach \x [count=\i from 1, remember=\i as \prev (initially 0)] in {0,1,0,0,0} {
			\node(chr\i a) [rectangle, draw, right =of chr\prev a] {$\x$};
		}
		\node(chr0b) [rectangle, draw] at (0,-1.4) {$0$};
		\node(chrb) [left =of chr0b] {$27$};
		\foreach \x [count=\i from 1, remember=\i as \prev (initially 0)] in {\alert{1},\alert{1},0,1,1} {
			\node(chr\i b) [rectangle, draw, right =of chr\prev b] {$\x$};
		}
	\end{tikzpicture}
\end{frame}

\begin{frame}{有界な文字の繰り返し:$\ShiftAnd$アルゴリズム}
	\begin{block}{改良した$\ShiftAnd$アルゴリズム}

		以下の更新式を加える

		\begin{equation*}
			D \leftarrow D | ((F - (D \& I)) \& \sim F)
		\end{equation*}

		\begin{itemize}
			\item $I$:繰り返しに対応する状態の始端を$1$としたビットマスク
			\item $F$:始端から$b-a+1$個目の状態を$1$としたビットマスク
		\end{itemize}
	\end{block}

	例 $P = AT(1,3)G(2,3)C$
	\begin{columns}
		\begin{column}{.33\textwidth}
			\begin{tikzpicture}[state/.style={circle, draw, minimum size=.7cm}, node distance=0cm]
				\node(chr0) [rectangle, draw] at (0,0) {$0$};
				\node(chr) [left =of chr0] {$I$};
				\foreach \x [count=\i from 1, remember=\i as \prev (initially 0)] in {0,0,0,1,0,0,1} {
					\node(chr\i) [rectangle, draw, right =of chr\prev] {$\x$};
				}
				\node(chr0a) [rectangle, draw] at (0,-.7) {$0$};
				\node(chra) [left =of chr0a] {$F$};
				\foreach \x [count=\i from 1, remember=\i as \prev (initially 0)] in {0,1,0,1,0,0,0} {
					\node(chr\i a) [rectangle, draw, right =of chr\prev a] {$\x$};
				}
			\end{tikzpicture}
		\end{column}
		\begin{column}{.66\textwidth}
			\only<1>{
				\begin{tikzpicture}[state/.style={circle, draw, minimum size=.7cm}, node distance=0cm]
					\node(d0) [rectangle, draw] at (0,0) {$0$};
					\node(d) [left =of d0] {$D$};
					\foreach \x [count=\i from 1, remember=\i as \prev (initially 0)] in {0,0,0,0,0,0,1} {
						\node(d\i) [rectangle, draw, right =of d\prev] {$\x$};
					}
					\node(is0) [rectangle, draw] at (0,-.7) {$0$};
					\node(is) [left =of is0] {$\& I$};
					\foreach \x [count=\i from 1, remember=\i as \prev (initially 0)] in {0,0,0,0,0,0,1} {
						\node(is\i) [rectangle, draw, right =of is\prev] {$\x$};
					}
					\node(dp0) [rectangle, draw] at (0,-1.4) {$0$};
					\node(dp) [left =of dp0] {$D\& I$};
					\foreach \x [count=\i from 1, remember=\i as \prev (initially 0)] in {0,0,0,0,0,0,1} {
						\node(dp\i) [rectangle, draw, right =of dp\prev] {$\x$};
					}
				\end{tikzpicture}
			}
			\only<2>{
				\begin{tikzpicture}[state/.style={circle, draw, minimum size=.7cm}, node distance=0cm]
					\node(d0) [rectangle, draw] at (0,0) {$0$};
					\node(d) [left =of d0] {$F$};
					\foreach \x [count=\i from 1, remember=\i as \prev (initially 0)] in {0,1,0,1,0,0,0} {
						\node(d\i) [rectangle, draw, right =of d\prev] {$\x$};
					}
					\node(is0) [rectangle, draw] at (0,-.7) {$0$};
					\node(is) [left =of is0] {$- (D\& I)$};
					\foreach \x [count=\i from 1, remember=\i as \prev (initially 0)] in {0,0,0,0,0,0,1} {
						\node(is\i) [rectangle, draw, right =of is\prev] {$\x$};
					}
					\node(dp0) [rectangle, draw] at (0,-1.4) {$0$};
					\node(dp) [left =of dp0] {$F-(D\& I)$};
					\foreach \x [count=\i from 1, remember=\i as \prev (initially 0)] in {0,1,0,0,1,1,1} {
						\node(dp\i) [rectangle, draw, right =of dp\prev] {$\x$};
					}
				\end{tikzpicture}
			}
			\only<3>{
				\begin{tikzpicture}[state/.style={circle, draw, minimum size=.7cm}, myLabel/.style={minimum width={width("(F-(DW I))W W F")}},node distance=0cm]
					\node(d0) [rectangle, draw] at (0,0) {$0$};
					\node(d) [myLabel, left =of d0] {$F-(D\& I)$};
					\foreach \x [count=\i from 1, remember=\i as \prev (initially 0)] in {0,1,0,0,1,1,1} {
						\node(d\i) [rectangle, draw, right =of d\prev] {$\x$};
					}
					\node(is0) [rectangle, draw] at (0,-.7) {$1$};
					\node(is) [myLabel, left =of is0] {$\& \sim F$};
					\foreach \x [count=\i from 1, remember=\i as \prev (initially 0)] in {1,0,1,0,1,1,1} {
						\node(is\i) [rectangle, draw, right =of is\prev] {$\x$};
					}
					\node(dp0) [rectangle, draw] at (0,-1.4) {$0$};
					\node(dp) [myLabel, left =of dp0] {$(F-(D\& I))\& \sim F$};
					\foreach \x [count=\i from 1, remember=\i as \prev (initially 0)] in {0,0,0,0,1,1,1} {
						\node(dp\i) [rectangle, draw, right =of dp\prev] {$\x$};
					}
				\end{tikzpicture}
			}
		\end{column}

	\end{columns}
\end{frame}
\end{document}
