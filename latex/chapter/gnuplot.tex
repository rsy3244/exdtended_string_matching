%! TEX root = ../main.tex
\documentclass[dvipdfmx,12pt,beamer]{standalone}
\ifstandalone
	%%%% 和文用(よくわかってないです) %%%%
\usepackage{bxdpx-beamer} %ナビゲーションシンボル(?)を機能させる
\usepackage{pxjahyper} %しおりの日本語対応
\usepackage{minijs} %和文メトリックの調整
\usepackage{latexsym}
\usepackage[deluxe,expert]{otf}
\renewcommand{\kanjifamilydefault}{\gtdefault} %和文規定をゴシックに(変えたければrenewcommand で調整)
%%%%%%%%%%

%%%% TikZ %%%%
\usepackage{tikz}
\usetikzlibrary{calc,decorations.pathreplacing,quotes,positioning,shapes,fit,arrows,backgrounds,tikzmark}
%%%%%%%%

%%%% ファイル分割用のパッケージ %%%%
\usepackage{standalone}
\usepackage{import}
%%%%%%%%

%%%% スライドの見た目 %%%%%
\usetheme{metropolis}
\usepackage{xcolor}
%\usefonttheme{professionalfonts}
\setbeamercolor{alerted text}{fg=mLightBrown!90!black}
\setbeamersize{text margin left=.75zw, text margin right=.75zw}
\setbeamertemplate{section in toc}[ball unnumbered]
\setbeamertemplate{subsection in toc}[square]
%%%%%%%%%%

%%%%% フォント基本設定 %%%%%
\usefonttheme{professionalfonts}
\usepackage[T1]{fontenc}%8bit フォント
%\usepackage{textcomp}%欧文フォントの追加
\usepackage[utf8]{inputenc}%文字コードをUTF-8
\usepackage{txfonts}%数式・英文ローマン体を Lxfont にする
%\usepackage{mathpazo}
\usepackage{bm}%数式太字
%%%%%%%%%%

%%%% その他ほしかったもの%%%%
\usepackage{amsmath,amssymb}
\usepackage{amsthm}
\usepackage{algorithm, algpseudocode}
\usepackage{enumitem}
\setlist[itemize]{label=\textbullet}

\newcommand{\func}[1]{\ensuremath\mathrm{#1}}
\newcommand{\Scatter}{\text{\scshape Scatter}}
\newcommand{\Gather}{\text{\textsc{Gather}}}
\newcommand{\Propagate}{\func{\text{\textsc{Propagate}}}}
\newcommand{\ShiftAnd}{\textbf{Shift-And}}
\newcommand{\BNDM}{\textbf{BNDM}}
\providecommand{\as}{\textasteriskcentered}
\providecommand{\pl}{\text{+}}
%%%%%%%%

%%%% Standaloneかどうかでimportのパスを切り替える %%%%
\providecommand{\ImportStandalone}[3]{
	\IfStandalone{
		\import{#2}{#3}
		}{
		\import{#1#2}{#3}
	}
}
%%%%%%%%

\title{Extended string matching}
\institute{北海道大学 情報知識ネットワーク研究室 M2}
\author{光吉 健汰}

\fi	

\begin{document}
\begin{frame}[fragile]{\LaTeX でグラフの描画}
	グラフの挿入で困ること
	\begin{itemize}
		\item データの変更があった場合に、グラフを作成し直す必要がある。
		\item グラフでのフォントと、文書のフォントが合わなくなる。
	\end{itemize}

	\begin{block}{pgfplots}
		\LaTeX のパッケージの一つ。PGFという描画エンジンを用いており、座標空間を描画できるaxis環境を提供している。
	\end{block}
	\begin{columns}
		\begin{column}{.45\textwidth}
			\begin{tikzpicture}[scale=.5]
				\begin{axis}[axis lines=center, xlabel=$x$, ylabel=$y$, legend pos=outer north east, ymax=2, ymin=-2]
					\addplot[thick, red, samples=100,domain=-2:2]{x^3};
					\addlegendentry{$y = x^3$}
					\addplot[thick, samples=100,domain=-2:2]{x^-1};
					\addlegendentry{$y = x^{-1}$}
				\end{axis}
			\end{tikzpicture}
		\end{column}
		\begin{column}{.54\textwidth}
			\begin{lstlisting}[language=tex]
\begin{tikzpicture}[scale=.6]
  \begin{axis}[axis lines=center, xlabel=$x$, ylabel=$y$, legend pos=outer north east, ymax=2, ymin=-2]
    \addplot[thick, samples=100,domain=-2:2]{x^3};
    \addlegendentry{$y = x^3$}
    \addplot[thick, samples=100,domain=-2:2]{x^-1};
    \addlegendentry{$y = x^2$}
  \end{axis}
\end{tikzpicture}
\end{lstlisting}
		\end{column}
	\end{columns}
\end{frame}

\begin{frame}[fragile]{csvファイルからのインポート}

	pgfplotsでは、csvファイルからグラフで描画したい列を取得することができる。

	\begin{columns}
		\begin{column}{.45\textwidth}
			\begin{tikzpicture}[scale=.5]
				\begin{axis}[date coordinates in=x, 
					xticklabel=\month-\day, name=g-covid,
					axis lines=center, xlabel=日付, ylabel=陽性数]
					\addplot[thick, red] %
						table [x={日付}, y={日陽性数}, col sep=comma]%
						{covid19_data_sapporo_2020_03.csv};
				\end{axis}
				\node [below =of g-covid] {{\tiny 例 札幌市の2020年3月のCOVID19陽性数\footnotemark}};
			\end{tikzpicture}
			
		\end{column}
		\begin{column}{.54\textwidth}
			\begin{lstlisting}[language=tex]
\begin{tikzpicture}[scale=.5]
  \begin{axis}[date coordinates in=x, 
    xticklabel=\month-\day,
    axis lines=center, xlabel=日付, ylabel=陽性数]
    \addplot[thick, red] %
      table [x={日付}, y={日陽性数}, col sep=comma]%
      {covid19_data_sapporo_2020_03.csv};
  \end{axis}
  \node [below =of g-covid] {{\tiny 例 札幌市の2020年3月のCOVID19陽性数\footnotemark}};
\end{tikzpicture}
\end{lstlisting}
		\end{column}
	\end{columns}
	\footnotetext[1]{\url{https://ckan.pf-sapporo.jp/dataset/covid_19_patients/resource/7dc6b374-ac73-4df4-b899-ea6c8cac3e32}}
\end{frame}

\end{document}

