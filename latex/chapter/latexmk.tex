
%! TEX root = ../main.tex
\documentclass[dvipdfmx,12pt,beamer]{standalone}
\ifstandalone
	%%%% 和文用(よくわかってないです) %%%%
\usepackage{bxdpx-beamer} %ナビゲーションシンボル(?)を機能させる
\usepackage{pxjahyper} %しおりの日本語対応
\usepackage{minijs} %和文メトリックの調整
\usepackage{latexsym}
\usepackage[deluxe,expert]{otf}
\renewcommand{\kanjifamilydefault}{\gtdefault} %和文規定をゴシックに(変えたければrenewcommand で調整)
%%%%%%%%%%

%%%% TikZ %%%%
\usepackage{tikz}
\usetikzlibrary{calc,decorations.pathreplacing,quotes,positioning,shapes,fit,arrows,backgrounds,tikzmark}
%%%%%%%%

%%%% ファイル分割用のパッケージ %%%%
\usepackage{standalone}
\usepackage{import}
%%%%%%%%

%%%% スライドの見た目 %%%%%
\usetheme{metropolis}
\usepackage{xcolor}
%\usefonttheme{professionalfonts}
\setbeamercolor{alerted text}{fg=mLightBrown!90!black}
\setbeamersize{text margin left=.75zw, text margin right=.75zw}
\setbeamertemplate{section in toc}[ball unnumbered]
\setbeamertemplate{subsection in toc}[square]
%%%%%%%%%%

%%%%% フォント基本設定 %%%%%
\usefonttheme{professionalfonts}
\usepackage[T1]{fontenc}%8bit フォント
%\usepackage{textcomp}%欧文フォントの追加
\usepackage[utf8]{inputenc}%文字コードをUTF-8
\usepackage{txfonts}%数式・英文ローマン体を Lxfont にする
%\usepackage{mathpazo}
\usepackage{bm}%数式太字
%%%%%%%%%%

%%%% その他ほしかったもの%%%%
\usepackage{amsmath,amssymb}
\usepackage{amsthm}
\usepackage{algorithm, algpseudocode}
\usepackage{enumitem}
\setlist[itemize]{label=\textbullet}

\newcommand{\func}[1]{\ensuremath\mathrm{#1}}
\newcommand{\Scatter}{\text{\scshape Scatter}}
\newcommand{\Gather}{\text{\textsc{Gather}}}
\newcommand{\Propagate}{\func{\text{\textsc{Propagate}}}}
\newcommand{\ShiftAnd}{\textbf{Shift-And}}
\newcommand{\BNDM}{\textbf{BNDM}}
\providecommand{\as}{\textasteriskcentered}
\providecommand{\pl}{\text{+}}
%%%%%%%%

%%%% Standaloneかどうかでimportのパスを切り替える %%%%
\providecommand{\ImportStandalone}[3]{
	\IfStandalone{
		\import{#2}{#3}
		}{
		\import{#1#2}{#3}
	}
}
%%%%%%%%

\title{Extended string matching}
\institute{北海道大学 情報知識ネットワーク研究室 M2}
\author{光吉 健汰}

\fi	

\begin{document}
\begin{frame}[fragile]{コンパイル自動化ツール Latexmk}
	\LaTeX のコンパイル時に困ること
	\begin{itemize}
		\item 処理系によっては直接pdfファイルを生成できない\\(dviファイルを作成する必要がある)。
		\item 参照解決のために何回もコンパイルする必要がある。
		\item Bib\TeX などのコンパイルの順序などに気を付けなければならない。
	\end{itemize}

	\begin{block}{Latexmk}
		コンパイル回数や、Bib\TeX などのコンパイルを自動で行う\\ビルドツール。
		\begin{lstlisting}
latexmk [option ..] [file ..]
\end{lstlisting}
	\end{block}
\end{frame}

\begin{frame}[fragile]{Latexmk 設定方法}
	ホームディレクトリに\fbox{.latexmkrc} を作成し、設定を記述することで、その設定を利用できる。
\begin{lstlisting}[frame=single, caption=.latexmkrc 設定例, basicstyle={\ttfamily\tiny}]
$clean_ext='bbl nav out snm aux sta synctex'; # -c オプションを入れた場合に削除する中間ファイル
$latex='platex -kanji=utf-8 -synctex=1 %S'; # latexの処理系の設定
$bibtex='pbibtex'; # bibtexの設定
$pdf_mode=3; # use dvipdf
$dvipdf='dvipdfmx %O -o %D %S'; # dviからpdfへの変換に使う処理系の設定
$max_repeat=5; # コンパイルを行う回数
$out_dir='build'; # 出力するディレクトリ
$pdf_previewer="open -ga /Applications/Skim.app"; # 作成したpdfファイルのビューアの設定
\end{lstlisting}

\end{frame}
\end{document}
